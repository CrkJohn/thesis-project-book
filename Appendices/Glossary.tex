\newcommand{\glossarydef}[4]{
    \newglossaryentry{#1}{
        text={\mbox{\textsc{#2}}},
        long={#3},
        name={#3\, \mbox{\textsc{(#2)}}},
        first={\textsl{#3}\, \mbox{\textsc{(#2)}}},
        firstplural={\textsl{\glsentrylong{#1}\glspluralsuffix}\, \mbox{(\textsc{\glsentrytext{#1}\glspluralsuffix})}},
        description={#4}
    }
}

\newcommand{\simpleglossarydef}[3]{
    \newglossaryentry{#1}{
        name={#2},
        long={#2},
        text={\textsl{#2}},
        first={\textsl{#2}},
        description={#3}
    }
}

% Permite que los terminos del glosario sean en cursiva tipo 'slanted'
\let\glsentrylongnostyle\glsentrylong
\renewcommand{\glsentrylong}[1]{\textsl{\glsentrylongnostyle{#1}}}

\glossarydef{nlp}{NLP}{Natural Language Processing}{Rama de la inteligencia artificial que lidia con la interacción entre computadores y humanos usando el lenguaje natural}

\glossarydef{machinel}{ML}{Machine Learning}{Informalmente ha sido definido como ``El campo de estudio que le da a computadores la habilidad de aprender sin ser explícitamente programados'', este tiene tres tipos de algoritmos de aprendizaje: aprendizaje supervisado, aprendizaje no--supervisado, y aprendizaje por refuerzo}

\simpleglossarydef{supervisedl}{Supervised Learning}{Aprendizaje supervisado, su objetivo es aprender un mapeo de unas entradas a unas salidas definidas}

\simpleglossarydef{unsupervisedl}{Unsupervised Learning}{Aprendizaje no-supervisado, su objetivo es obtener patrones estructurados de una serie de datos no estructurados}

\simpleglossarydef{reinforcedl}{Reinforcement Learning}{Aprendizaje de refuerzo, es el aprendizaje de que acción se debe tomar de forma que se logre una señal de recompensa lo mas alto posible}

\simpleglossarydef{feedforward}{Feedforward}{Es un termino que describe típicamente un elemento o un camino dentro de un sistema de control que pasa una señal de control de una fuente externa. En la inteligencia artificial este termino no difiere en significado}

\simpleglossarydef{clustering}{Clustering}{Es la tarea de agrupar conjuntos de objetos de manera que este en grupos de mismo tipo}

\glossarydef{deepl}{DL}{Deep Learning}{Es un subcampo de Machine Learning que se usa algoritmos inspirados por la estructura y función del cerebro que son llamadas Redes Neuronales Artificiales (\'o Artificial Neural Networks en ingl\'es)}

\simpleglossarydef{dataintegration}{Data Integration}{Para acceder a múltiples y diversas fuentes de información}

\simpleglossarydef{linkanalisys}{Link Analysis}{Para visualizar asociaciones y relaciones criminales y terroristas}

\simpleglossarydef{softwareagents}{Software Agents}{Para el monitoreo, obtención, análisis y actuación sobre la información}

\simpleglossarydef{textmining}{Text Mining}{Búsqueda sobre terabytes de información en documentos, paginas web y correos electrónicos}

\simpleglossarydef{datamining}{Data Mining}{Proceso de descubrir automáticamente información útil en repositorios grandes de datos}

\glossarydef{ann}{ANN}{Artificial Neural Network}{Para predecir la probabilidad de crímenes y nuevos ataques terroristas}

\simpleglossarydef{mlalgorithms}{Machine Learning Algorithms}{Para extraer perfiles de perpetradores y mapas gráficos de crímenes}

\glossarydef{kbs}{KBS}{Knowledge Based Systems}{Sistemas Basados en Conocimiento, su objetivo es abstraer conocimiento de un experto de un área en una representación dentro de un computador}

\glossarydef{ai}{AI}{Artificial Intelligence}{Inteligencia Artificial, área de las ciencia de la computación que se encarga de la creación de maquinas inteligentes que tienen un comportamiento que se asemeja a tener inteligencia}

\simpleglossarydef{anomalydetectionsys}{Anomaly Detection Systems}{Sistemas de Detección de Anomalías, son sistemas con el objetivo de encontrar datos atípicos dentro de un conjunto de datos, típicamente se tiene un entrenamiento previo que muestra el comportamiento típico de los datos para luego encontrar las anomalías}

\glossarydef{som}{SOM}{Self-organizing Map}{Mapas auto-organizados, son redes neuronales especializadas para el clustering de datos semejantes según una función de distancia, da como resultado un mapa discreto uni- o bi-dimensional}

\glossarydef{svm}{SVM}{Support Vector Machine}{Maquinas de Soporte Vectorial, son modelos de aprendizaje supervisado usados para el problema de clasificación y análisis de regresión}

\glossarydef{bow}{BoW}{Bag of Words}{Es una representación simplificada usada en procesamiento de lenguaje natural. En este modelo, un texto es representado como una bolsa (multiconjunto) de sus palabras}

\glossarydef{tfidf}{TF-IDF}{Term Frequency -- Inverse Document Frequency}{Frecuencia de términos -- Frecuencia inversa de documento, es una estadística numérica que esta diseñada para reflejar que tan importante es una palabra a un documento dentro de un corpus}

\glossarydef{osint}{OSINT}{Open Source Intelligence}{Disciplina responsable de la adquisición, procesamiento y posterior transformación en inteligencia de información obtenida de fuentes públicas como prensa, radio, televisión, internet, informes de diferentes sectores y, en general, cualquier recurso de acceso público (Tomado de \cite{osint})}

\glossarydef{knowledgebase}{KB}{Knowledge Base}{Base de conocimiento, es una tecnología usada para almacenar información estructurada y no-estructurada usada por un sistema de computo}

\glossarydef{inferenceengine}{IE}{Inference Engine}{Motor de inferencia, es un componente del sistema que aplica reglas lógicas a la base de conocimiento para deducir información nueva}

\simpleglossarydef{expertsystems}{Expert System}{Sistemas expertos, son sistemas de computo que emulan la abeldad de toma de decisiones de un humano experto}

\glossarydef{rulebasedsys}{RBS}{Rule--based Systems}{Sistemas basados en reglas, son sistemas expertos que en su base de conocimiento contienen reglas para crear nuevas inferencias}

\glossarydef{workingmemory}{WM}{Working Memory}{Memoria de trabajo, es donde se encuentra el estado actual del sistema antes y después de haber ejecutado reglas de un sistema basado en reglas, y es donde se encuentran las inferencias hechas por el motor de inferencia}

\simpleglossarydef{deductivesys}{Deductive Systems}{Sistemas deductivos, son sistemas basados en conocimiento que en base al estado de la memoria de trabajo y a las reglas presentes en la base de conocimiento, agregan nuevas inferencias a la memoria de trabajo}

\simpleglossarydef{reactivesys}{Reactive Systems}{Sistemas reactivos, son sistemas basados en conocimiento que en base al estado de la memoria de trabajo y a las reglas presentes en la base de conocimiento, agregan o eliminan inferencias de la memoria de trabajo}

\simpleglossarydef{coursera}{Coursera}{Sitio web de cursos de aprendizaje en \href{https://www.coursera.org}{https://www.coursera.org}}

\simpleglossarydef{udemy}{Udemy}{Sitio web de cursos de aprendizaje en \href{https://www.udemy.com}{https://www.udemy.com}}

\simpleglossarydef{dataset}{Dataset}{Conjunto de datos, en el campo de Inteligencia Artificial generalmente se refiere a conjuntos públicos o de acceso limitado de donde es posible obtener información útil con sistemas de computo}

\glossarydef{lea}{LEA}{Law Enforcement Agency}{Agencia de cumplimiento de la ley, son agencias gubernamentales encargadas de que la población respete la ley}

\simpleglossarydef{mmh}{Maximum Margin Hyperplanes}{Hiperplanos que permiten separar datos en espacios de alta dimensionalidad con un margen asociado para separarlos}

\glossarydef{lstm}{LSTM}{Long Short Term Memory}{Red neuronal recurrente que tiene memoria Largo y Corto Plazo, son un tipo de redes neuronales recurrentes que son capaces de aprender dependencias de largo plazo}

\glossarydef{bilstm}{Bi-LSTM}{Bidirectional Long Short Term Memory}{Red neuronal LSTM bidireccional, que permite el uso de secuencias de palabras futuras y pasadas para una mejor clasificacion de la actual}

\glossarydef{rnn}{RNN}{Recurrent Neural Network}{Redes que recuerdan salidas y entradas pasadas de datos, de forma que decisiones futuras respecto a una clasificación pueden tener un mejor resultado}

\simpleglossarydef{namedent}{Named Entities}{Son típicamente palabras que denotan personas particulares u organizaciones, pero no son sustantivos}

\simpleglossarydef{starspace}{StarSpace}{Es un modelo neuronal de propósito general para el aprendizaje eficiente de embeddings}

\simpleglossarydef{overfitting}{Overfitting}{El modelo esta demasiado acoplado a la entrada por lo que para un modelo predictivo dará resultados solo condicionados para la entrada de entrenamiento, pero tendrá mal desempeño en cualquier otro conjunto de prueba}

\simpleglossarydef{underfitting}{Underfitting}{El modelo predice pobremente la salida correcta en base a la entrada, de forma que no da buenos resultados para cualquier conjunto de entrenamiento o de pruebas}

\simpleglossarydef{bias}{Bias}{Es la parcialidad de un modelo, donde entre mayor sea peor sera la predicción del modelo respecto a cualquier dato que se le de como entrada}

\simpleglossarydef{datascience}{Data Science}{Es una campo multidisciplinario que usa metodos, procesos, algoritmos y sistemas cientificos para extraer conocimiento y revelaciones de datos estructurados y no estructurados}

\simpleglossarydef{tensorflow}{TensorFlow}{TensorFlow es una plataforma de codigo abierto para machine learning. Tiene un ecosistema comprensible y flexible de herramientas, librerias y recursos de la comunidad que le permite a investigadores impulsar el estado del arte en machine learning y desarrolladores puede construir y desplegar facilmente aplicaciones con Machine Learning incorporado}

\glossarydef{crisp-dm}{CRISP-DM}{Cross Industry Standard Process for Data Mining}{Proporciona una descripción normalizada del ciclo de vida de un proyecto estándar de análisis de datos. El modelo CRISP-DM cubre las fases de un proyecto, sus tareas respectivas, y las relaciones entre estas tareas}

\simpleglossarydef{corpus}{Corpus}{Es un conjunto grande y estructurado de textos, compuesto principalmente por documentos que contienen terminos (palabras) ordenados}

\glossarydef{roc}{ROC}{Receiver-Operating Characteristic}{\todo[inline]{Pendiente}}

\glossarydef{auc}{AUC}{Area Under the Curve}{\todo[inline]{Pendiente}}

\glossarydef{mlp}{MLP}{Multilayer Perceptrons}{\todo[inline]{Pendiente}}

\glossarydef{dag}{DAG}{Directed Acyclic Graph}{\todo[inline]{Pendiente}}