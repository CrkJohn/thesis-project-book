% Appendix A

\chapter{Información general del proyecto} % Main appendix title

\label{appendix:projectinfo} % For referencing this appendix elsewhere, use \ref{AppendixA}

\section{Repositorio del proyecto}
El repositorio principal que contiene todos el Software, documentación y este documento se encuentran en GitHub. \\

\begin{tcolorbox}[colback=gray!5!white,
  colframe=black!75!white,
  title=Repositorio de GitHub]
  Hiper-vinculo al repositorio: \\
  \hspace*{1em} \href{https://github.com/aanzolaavila/thesis-project-repository.git}{https://github.com/aanzolaavila/thesis-project-repository.git} \\ [1em]

  Para clonar el repositorio es necesario ejecutar el comando en la Terminal (o bien \textit{Git Bash} en Windows):
  \begin{minted}[breaklines,fontsize=\small]{bash}
git clone --recurse-submodules https://github.com/aanzolaavila/thesis-project-repository
  \end{minted}
\end{tcolorbox}

\section{Software del proyecto}
Cada software esta en repositorios independientes dentro de GitHub, donde tambien se puede encontrar una documentación de como ejecutarlos.

\subsection{Mapas autoorganizados}
\todo[inline]{Por hacer}

\subsection{Modelos de software}
\todo[inline]{Por hacer}

\subsection{Extractor de tweets de \textsc{Archive}}
\todo[inline]{Por hacer}

\subsection{Consumidor de Twitter}
\todo[inline]{Por hacer}

\subsection{StarSpace}
\todo[inline]{Por hacer}
