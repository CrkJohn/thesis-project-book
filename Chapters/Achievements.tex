\chapter{Logros y productos del proyecto} % Main chapter title

\label{chap:achievements} % For referencing the chapter elsewhere, use \ref{Chapter1}

Entre los logros y productos obtenidos durante la ejecución de este proyecto de grado se encuentran los siguientes:

\begin{enumerate}
\item Entendimiento de las generalidades de \gls{datascience}:
  \begin{itemize}
  \item Tipos de \glsentrylong{machinel}.
  \item Sistemas de detección de anomalías (\glsentrylong{anomalydetectionsys}).
  \item Diferentes modalidades de \gls{clustering}.
  \end{itemize}
\item Identificación de modelos de \gls{nlp} aplicables para el perfilado de cibercriminales.
\item Entendimiento de los modelos de clasificación y \gls{clustering}:
  \begin{itemize}
  \item Clasificador de Na\"ive Bayes.
  \item Maquinas de soporte vectorial (\glsentrylong{svm}).
  \item Mapas autoorganizados (\glsentrylong{som}).
  \end{itemize}
\item Entendimiento de los modelos utilizados en \gls{nlp}:
  \begin{itemize}
  \item Predicción de etiquetas con modelos de regresión lineal.
  \item Reconocimiento de \gls{namedent}.
  \item Uso de \emph{embeddings} generados con \gls{starspace} para los $k$ textos mas similares.
  \end{itemize}
\item Propuesta de modelos de \gls{nlp} para el perfilado de cibercriminales:
  \begin{itemize}
  \item Modelo de predicción de hashtags de Twitter con modelos lineales junto con su implementación.
  \item Modelo de reconocimiento de \gls{namedent} con redes \glsentrylong{lstm}.
  \item Modelo de \gls{clustering} en redes \glsentrytext{som} con \emph{embeddings} de \gls{starspace}.
  \end{itemize}
\end{enumerate}
