\chapter{Conclusiones y trabajos futuros}

\label{chap:conclusions}

\section{Conclusiones}
\todo{Revisar bien y comparar con los anteriores}
\begin{itemize}
\item El procesamiento de lenguaje natural (NLP) aporta un conjunto de metodologías y técnicas que pueden apoyar labores de ciberinteligencia.

\item El éxito en la aplicación de modelos de aprendizaje automático depende en gran medida de la calidad de los datos de entrenamiento.
\end{itemize}

\section{Trabajos futuros}
\todo{Revisar bien}
\begin{itemize}
\item Utilizar el meta-modelo de \gls{som} (\cref{subsec:SOM}) como un cuarto modelo para el clustering de textos representados con embeddings pre-entrenados o generados con \gls{starspace}.
  
\item Utilizar metodologías de visualización de las redes \gls{bilstm} como la visualización expuesta en \cite{madsen2019visualizing} con el fin de depurar su ejecución.
  
\item Desarrollo de un \emph{dashboard} para la ayuda de visualización de un agente de seguridad del estado a desempeñar su tarea de perfilado de cibercriminales.
  
\item Desarrollo de técnicas avanzadas de obtención de información de redes sociales, tanto de redes sociales populares como impopulares, de manera similar al desarrollado con \gls{osint} en \cite{osint}.

\item Desarrollar un dashboard que integre los resultados obtenidos de los tres modelos.

\item Extender la aplicación de los modelos desarrollados al lenguaje español (Colombia).
\end{itemize}
