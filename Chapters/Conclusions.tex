\chapter{Conclusiones y trabajos futuros}

\label{chap:conclusions}

\section{Conclusiones}
\begin{itemize}
\item Se investigaron diferentes metodologías de \gls{nlp} y \gls{datascience} para la tarea de perfilado de cibercriminales por medio de información de fuentes abiertas.
\item Es necesario probar las metodologías propuestas con información obtenida de fuentes abiertas que este validada de forma que el entrenamiento de ellos sean efectivos en la tarea.
\item Se logro un obtener una vista amplia de las posibles aplicaciones de \gls{nlp} aplicados a contextos de ciber espacio.
\end{itemize}

\section{Trabajos futuros}
\begin{itemize}
\item Utilizar el meta-modelo de \gls{som} (\sectionref{subsec:SOM}) como un cuarto modelo para el clustering de textos representados con embeddings pre-entrenados o generados con \gls{starspace}.
\item Utilizar metodologías de visualización de las redes \gls{bilstm} como la visualización expuesta en \cite{madsen2019visualizing} con el fin de depurar su ejecución.
\item Desarrollo de un \emph{dashboard} para la ayuda de visualización de un agente de seguridad del estado a desempeñar su tarea de perfilado de cibercriminales.
\item Desarrollo de técnicas avanzadas de obtención de información de redes sociales, tanto de redes sociales populares como impopulares, de manera similar al desarrollado con \gls{osint} en \cite{osint}.
\end{itemize}
