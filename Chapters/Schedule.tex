\chapter{Cronograma} % Main chapter title

\label{ch:Schedule} % For referencing the chapter elsewhere, use \ref{Chapter1} 

%----------------------------------------------------------------------------------------

% \thispagestyle{empty}
\setcounter{taskcounter}{0}
\begin{figure}[h!]
  \begin{tikzpicture}
    \GanttHeader{\textwidth}{2ex}{5cm}{16}
    \Taske{0}{5}{Entendimiento de \glsentrytext{osint}}
    \Taske{1}{9}{Estudio de literatura de \gls{nlp}}
    \Taske{2}{6}{Investigación del Estado del Arte}
    \Taske{6}{9}{Desarrollo de propuesta}
    \Taske{6}{9}{Implementación de software}
    \Taske{8}{6}{Desarrollo del curso de \gls{nlp}}
    \Taske{14}{2}{Generación de documentación técnica}
  \end{tikzpicture}
  \caption{Diagrama Gantt de actividades de 1\textsuperscript{er} periodo.}
  \label{fig:gantt1}
\end{figure}


\setcounter{taskcounter}{0}
\begin{figure}[h!]
  \begin{tikzpicture}
    \GanttHeader{\textwidth}{2ex}{5cm}{16}

    \Taske{0}{3}{Terminación de software P1}
    \Taske{0}{8}{Profundización en \gls{nlp}}
    \Taske{3}{3}{Desarrollo de técnicas de \glsentrytext{osint}}
    \Taske{6}{3}{Visualización de métodos}
    \Taske{9}{3}{Desarrollo de métodos adicionales}
    \Taske{12}{3}{Pruebas de concepto}
    \Taske{14}{2}{Generación de documentación técnica}
  \end{tikzpicture}
  \caption{Diagrama Gantt de actividades de 2\textsuperscript{do} periodo (Planeado).}
  \label{fig:gantt2}
\end{figure}

\vspace{1cm}

\newcounter{tablecountone}
\newcounter{tablecounttwo}
\setcounter{tablecountone}{1}
\setcounter{tablecounttwo}{0}
\newcommand{\rowt}[1]{\addtocounter{tablecounttwo}{1}{\thetablecountone.\thetablecounttwo} & {#1.} \\ \hline}
\newcommand{\tablenextlevel}{\addtocounter{tablecountone}{1}\setcounter{tablecounttwo}{0}}

\begin{table}[h!]
\begin{center}
\begin{tabular}{|c p{.975\textwidth}|} \hline
  \textbf{\#} & \textbf{Descripción} \\ \hline \hline
  
  \rowt{Entendimiento proyecto de \emph{Inteligencia de fuentes abiertas para el contexto colombiano} \cite{osint}}
  \rowt{Estudio de la literatura de \gls{nlp}}
  \rowt{Investigación del Estado del Arte}
  \rowt{Desarrollo de la propuesta de solución}
  \rowt{Implementación de solución y pruebas}
  \rowt{Realización de curso de \gls{nlp} de \emph{National Research University Higher School of Economics} en \gls{coursera}}
  \rowt{Preparación final de documento de libro de proyectos del primer periodo}
  % ================ SEGUNDO PERIODO ================
  \tablenextlevel
  \rowt{Terminación y mejora de software desarrollado en el primer periodo}
  \rowt{Profundización de metodologías aplicadas de \gls{nlp}}
  \rowt{Desarrollo de nuevas técnicas para la obtención de información de fuentes abiertas como también de identificar nuevas fuentes de información que tengan que ver con terrorismo}
  \rowt{Desarrollo de un \emph{dashboard} detallado que ayuden un agente de ciberinteligencia realizar sus labores}
  \rowt{Desarrollo de métodos adicionales para el perfilamiento de cibercriminales}
  \rowt{Realizar pruebas demostrativas que permitan visualizar la obtención de información de las fuentes abiertas en tiempo real}
  \rowt{Generación de documentación técnica final}
\end{tabular}
\end{center}
\caption{Descripción del cronograma de actividades.}
\label{table:schedule}
\end{table}
