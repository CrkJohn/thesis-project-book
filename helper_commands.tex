% GanttHeader setups some parameters for the rest of the diagram
% #1 Width of the diagram
% #2 Width of the space reserved for task numbers
% #3 Width of the space reserved for task names
% #4 Number of months in the diagram
% In addition to these parameters, the layout of the diagram is influenced
% by keys defined below, such as y, which changes the vertical scale
\def\GanttHeader#1#2#3#4{%
  \pgfmathparse{(#1-#2-#3)/#4}
  \tikzset{y=7mm, task number/.style={left},
    task description/.style={text width=#3,  right, draw=none,
      xshift=#2,
      minimum height=2em},
    gantt bar/.style={draw=black, fill=red!50!black},
    help lines/.style={draw=black!30, dashed},
    x=\pgfmathresult pt
  }
  \def\totalmonths{#4}
  \node (Header) [task description] at (0,0) {\textbf{\large Actividades}};
  \begin{scope}[shift=($(Header.south east)$)]
    \foreach \x in {1,...,#4}
    \node[above] at (\x,0) {\footnotesize\x};
  \end{scope}
}

% This macro adds a task to the diagram
% #1 Number of the task
% #2 Task's name
% #3 Starting date of the task (month's number, can be non-integer)
% #4 Task's duration in months (can be non-integer)
\def\Task#1#2#3#4{%
  \node[task number] at ($(Header.west) + (0, -#1)$) {#1};
  \node[task description] at (0,-#1) {#2};
  \begin{scope}[shift=($(Header.south east)$)]
    \draw (0,-#1) rectangle +(\totalmonths, 1);
    \foreach \x in {1,...,\totalmonths}
    \draw[help lines] (\x,-#1) -- +(0,1);
    \filldraw[gantt bar] ($(#3, -#1+0.2)$) rectangle +(#4,0.6);
  \end{scope}
}

\newcounter{taskcounter}
\setcounter{taskcounter}{0}
\newcommand{\Taske}[3]{\addtocounter{taskcounter}{1}\Task{\thetaskcounter}{\footnotesize #3}{#1}{#2}}

% ================================================================

%table helpers
\newcolumntype{L}[1]{>{\raggedright\let\newline\\\arraybackslash\hspace{0pt}}m{#1}}
\newcolumntype{C}[1]{>{\centering\let\newline\\\arraybackslash\hspace{0pt}}m{#1}}
\newcolumntype{R}[1]{>{\raggedleft\let\newline\\\arraybackslash\hspace{0pt}}m{#1}}

% ================================================================

% Define some commands to keep the formatting separated from the conten

\newcommand{\norm}[1]{\left\lVert#1\right\rVert}
\newcommand{\keyword}[1]{\textbf{#1}}
\newcommand{\tabhead}[1]{\textbf{#1}}
\newcommand{\code}[1]{\texttt{#1}}
\newcommand{\file}[1]{\texttt{\bfseries#1}}
\newcommand{\option}[1]{\texttt{\itshape#1}}